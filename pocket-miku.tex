\documentclass[titlepage]{article}

%
% Notes on the Gakken NSX-39 ``Pocket Miku''
% Copyright (C) 2014  Matthew Skala
%
% This program is free software: you can redistribute it and/or modify
% it under the terms of the GNU General Public License as published by
% the Free Software Foundation, version 3.
%
% This program is distributed in the hope that it will be useful,
% but WITHOUT ANY WARRANTY; without even the implied warranty of
% MERCHANTABILITY or FITNESS FOR A PARTICULAR PURPOSE.  See the
% GNU General Public License for more details.
%
% You should have received a copy of the GNU General Public License
% along with this program.  If not, see <http://www.gnu.org/licenses/>.
%
% Matthew Skala
% http://ansuz.sooke.bc.ca/
% mskala@ansuz.sooke.bc.ca
%

\usepackage{fontspec}
\usepackage[margin=1.25in]{geometry}
\usepackage{tikz}
\usepackage{xltxtra}

% colors
\definecolor{darkgreen}{rgb}{0,0.35,0}
\definecolor{purplish}{rgb}{0.4,0,0.6}

\usepackage[letterpaper,breaklinks,bookmarks,plainpages=false,
   colorlinks,citecolor=darkgreen,linkcolor=purplish]{hyperref}

\defaultfontfeatures{Mapping=tex-text,Path=./}
\newfontfamily\mincho{TsukurimashouMinchoPS}

\newcommand{\futatsutomoe}{{\fontspec[RawFeature=+ornm]{TsukurimashouMinchoPS}B}}

\renewcommand{\labelitemi}{{\fontspec[RawFeature=+ornm]{TsukurimashouMinchoPS}C}}

\title{Notes on the Gakken NSX-39 ``Pocket Miku''}
\author{Matthew Skala,
}

%%%%%%%%%%%%%%%%%%%%%%%%%%%%%%%%%%%%%%%%%%%%%%%%%%%%%%%%%%%%%%%%%%%%%%%%

\begin{document}
\maketitle

%%%%%%%%%%%%%%%%%%%%%%%%%%%%%%%%%%%%%%%%%%%%%%%%%%%%%%%%%%%%%%%%%%%%%%%%

\section{Introduction}

The Gakken NSX-39 ``Pocket Miku'' is a music synthesizer included with the
April 2014 issue of the Japanese magazine {\mincho 「大人の科学」}
(\textit{Otona no Kagaku}, ``Science for Adults''), which usually includes
some kind of science project or gadget in each issue.  The NSX-39 is a small
electronic device resembling an external dialup modem circa 1995.  It can be
used standalone (powered by three AAA batteries and controlled with the
built-in stylophone), or connected to a computer with USB (requires a
micro-B USB cable, not included with the device).  It includes a
stylophone-type 27-note musical keyboard with a continuous-tone ``ribbon
controller'' section; a more or less complete General MIDI synthesizer
module with a class-compliant USB-MIDI interface (no device-specific driver
required on the computer); and a singing synthesizer based on Yamaha
``Vocaloid'' synthesis technology, featuring the famous ``Hatsune Miku''
synthesized voice.  Output is through a built-in piezo speaker (mono, and
almost no bass) or a standard 1/8'' headphone jack socket (stereo, full
range).

This document is intended to provide helpful information and notes for
users of the NSX-39, especially those outside Japan.

\subsection{Important warnings}

Please note that \textbf{the booklet that comes with the device is entirely
in Japanese}.  If you cannot read Japanese, then the booklet that comes with
it will be of little use to you.  This document exists partly to fill that
gap.  The controls on the device, however, are labelled in English, and one
of the optional stickers that comes with it is an English-language quick
reference to the controls.

Similarly, \textbf{the ``vocaloid'' synthesizer is designed to sing only in
Japanese}, and the manufacturer's online service for programming it
functions only in Japanese.  Programming it to sing in any language other
than Japanese is tricky (but this document and the associated software
should help), and the results will be so heavily Japanese-accented as
to be barely comprehensible.

Furthermore, note that because the device was only intended for sale within
Japan, if you are outside Japan then \textbf{you are on your own regarding
regulatory compliance issues} like RoHS, RF interference, safety labelling,
environmental fees, and any other requirements your country may impose on
electronic devices.  The litany of safety certifications, warning notices,
and test-lab logos usually found in the manuals of internationally-sold
electronic devices, is conspicuously absent from the Pocket Miku's manual. 
Depending on where you live, it may not be technically legal for you to
re-sell (or maybe even to use) this device.  And if it causes an
electrical fire, your insurance company may give you a hard time.

Finally, note that there is a long complicated license agreement on the
inside back cover of the manual, all in Japanese legalese, which I imagine
the corporations involved would claim is binding on you whether you can
understand it or not.  I think it says you're not allowed to make anime
porn videos of the ``Hatsune Miku'' character, which of course hasn't
stopped anybody\footnote{She's sixteen, you perverts!}; but I'm a little
more concerned about whether it might attempt to place \textbf{copyright
claims on music you make}.  So far I haven't deciphered it far enough to be
sure.

\subsection{This document, and contact information}

An up-to-date version of this document should be available from
\url{http://ansuz.sooke.bc.ca/pocket-miku.pdf}.  I expect to improve it from
time to time, so if the date on the front cover of your copy is not recent,
you may want to check for a new one.

Source code for this document, and associated software, is available from
\url{https://github.com/mskala/pocket-miku}.  To compile the document you
will need \XeTeX\ and the font {\mincho Tsukurimashou Mincho PS}, available
from \url{http://tsukurimashou.sourceforge.jp/}.

I welcome suggestions and information for inclusion in this document. 
Contact me by email at
\href{mailto:mskala@ansuz.sooke.bc.ca}{mskala@ansuz.sooke.bc.ca}.  Other
places of interest the synthesists where I can be found include the
Muffwiggler Forum as mskala (but please, use email in preference to
private-messaging there); on Twitter as
\href{https://twitter.com/mattskala}{mattskala}; and on Soundcloud as
\href{https://soundcloud.com/matt-skala}{matt-skala}.

\section{Front-panel controls}

\section{The stylophone keyboard}

\section{Programming the Vocaloid}

\section{MIDI synthesis}

The NSX-39's MIDI synthesizer seems to be somewhere between General MIDI
(original, not ``2'') and Yamaha's XG MIDI extension.  It supports many, but
apparently not all, of the XG extensions to General MIDI.

If you wish to use MIDI percussion then \textbf{the percussion channel must
be set to drum kit bank MSB 127, LSB 0}.  This is the power-on default for
MIDI channel 10, the default percussion channel; it is also the standard
drum kit for XG MIDI.  However, the Rosegarden sequencer (and possibly
others) will, unless specificaly configured otherwise, automatically
reinitialize percussion channels to MSB 0, LSB 1, which would be the GM
standard drum kit; and on that setting percussion will produce no sound.  I
have not tried every single one of the more than 16000 possible settings,
but it appears that the NSX-39 contains no other drum kits except MSB 127,
LSB 0.

The NSX-39's XG standard drum kit is a superset of the General MIDI standard
kit, containing some extra sounds.  Here is the complete list as seen in the
XG specification:

\begin{tabular}{cc}
  MIDI note \# & name \\ \hline
  13 & Surdo Mute \\
  14 & Surdo Open \\
  15 & Hi Q \\
  16 & Whip Slap \\
  17 & Scratch H \\
  18 & Scratch L \\
  19 & Finger Snap \\
  20 & Click Noise \\
  21 & Metronome Click \\
  22 & Metronome Bell \\
  23 & Seq Click L \\
  24 & Seq Click H \\
  25 & Brush Tap \\
  26 & Brush Swirl \futatsutomoe \\
  27 & Brush Slap \\
  28 & Brush Tap Swirl \futatsutomoe \\
  29 & Snare Roll \futatsutomoe \\
  30 & Castanet \\
  31 & Snare Soft \\
  32 & Sticks \\
  33 & Kick Soft \\
  34 & Open Rim Shot \\
  35 & Kick Tight \\
  36 & Kick \\
  37 & Side Stick \\
  38 & Snare \\
  39 & Hand Clap \\
  40 & Snare Tight \\
  41 & Floor Tom L \\
  42 & Hi-Hat Closed \\
  43 & Floor Tom H \\
  44 & Hi-Hat Pedal \\
  45 & Low Tom \\
  46 & Hi-Hat Open \\
  47 & Mid Tom L \\
  48 & Mid Tom H
\end{tabular}\qquad
\begin{tabular}{cc}
  MIDI note \# & name \\ \hline
  49 & Crash Cymbal 1 \\
  50 & High Tom \\
  51 & Ride Cymbal 1 \\
  52 & Chinese Cymbal \\
  53 & Ride Cymbal Cup \\
  54 & Tambourine \\
  55 & Splash Cymbal \\
  56 & Cowbell \\
  57 & Crash Cymbal 2 \\
  58 & Vibraslap \\
  59 & Ride Cymbal 2 \\
  60 & Bongo H \\
  61 & Bongo L \\
  62 & Conga H Mute \\
  63 & Conga H Open \\
  64 & Conga L \\
  65 & Timbale H \\
  66 & Timbale L \\
  67 & Agogo H \\
  68 & Agogo L \\
  69 & Cabasa \\
  70 & Maracas \\
  71 & Samba Whistle H \futatsutomoe \\
  72 & Samba Whistle L \futatsutomoe \\
  73 & Guiro Short \\
  74 & Guiro Long \futatsutomoe \\
  75 & Claves \\
  76 & Wood Block H \\
  77 & Wood Block L \\
  78 & Cuica Mute \\
  79 & Cuica Open \\
  80 & Triangle Mute \\
  81 & Triangle Open \\
  82 & Shaker \\
  83 & Jingle Bells \\
  84 & Bell Tree
\end{tabular}

The General MIDI sounds are numbers 35 to 81.  Sounds marked \futatsutomoe\
can be extended arbitrarily; they function like notes in ordinary
non-percussion channels, sounding from a Note ON message to the
corresponding Note OFF message.  Such sounds may cause trouble with some
computer sequencers (e.g.\ Rosegarden again), which sometimes send a Note ON
message and never cancel it in percussion channels.

%%%%%%%%%%%%%%%%%%%%%%%%%%%%%%%%%%%%%%%%%%%%%%%%%%%%%%%%%%%%%%%%%%%%%%%%

\end{document}